%% LyX 2.2.3 created this file.  For more info, see http://www.lyx.org/.
%% Do not edit unless you really know what you are doing.
\documentclass[12pt]{article}
\usepackage[latin9]{luainputenc}
\usepackage{float}
\usepackage{amsmath}
\usepackage{graphicx}
\usepackage{setspace}
\onehalfspacing

\makeatletter

%%%%%%%%%%%%%%%%%%%%%%%%%%%%%% LyX specific LaTeX commands.
%% A simple dot to overcome graphicx limitations
\newcommand{\lyxdot}{.}


%%%%%%%%%%%%%%%%%%%%%%%%%%%%%% User specified LaTeX commands.
\textwidth 15.5cm \oddsidemargin 0cm \topmargin -1cm \textheight
23cm \footskip 1.0cm \usepackage{epsfig}
\usepackage{graphicx}\usepackage{psfrag}\usepackage[numbered, framed]{matlab-prettifier}\usepackage{pstcol}\def\n{\noindent}
\def\u{\underline}
\def\hs{\hspace}
\newcommand{\thrfor}{.^{\displaystyle .} .}
\newcommand{\bvec}[1]{{\bf #1}}
\usepackage{layout}

\makeatother

\usepackage{listings}
\lstset{style={Matlab-editor}}
\begin{document}
\noindent \rule{15.7cm}{0.5mm}
\begin{center}
\textbf{ENGINEERING TRIPOS PART II A} 
\par\end{center}

\vspace{0.5cm}
 \textbf{EIETL \hfill{}MODULE EXPERIMENT 3F1} \vspace{0.5cm}
 
\begin{center}
\textbf{FLIGHT CONTROL}\\
\textbf{FULL TECHNICAL REPORT}\\
\textbf{\hfill{}}\\
\textbf{Name: Sushant Achawal}\\
\textbf{\hfill{}}\\
\textbf{ College: Emmanuel College}\\
\textbf{\hfill{}}\\
\par\end{center}

\rule{15.7cm}{0.5mm}
\begin{abstract}
Control systems are important all over Engineering, from the inverted
pendulum to regulating levels during drug administration. These systems
are often difficult to measure without uncertainty so methods must
be developed to understand their behaviour as well as being able to
control them to a desired result. In this experiment, Control Theory
is investigated in the context of flight control through a flight
simulator software in MATLAB. Systems are linearised and transfer
funtions are modelled using techniques such as Nyquist diagrams and
Bode plots. The varied behaviours of different transfer functions
are evaluated. A technique is developed to model the user as a controller
and then effective PID control is investigated using Ziegler-Nichols
tuning rules. Various themes of control are explored including the
discretisation, unstable poles as well as a comparison of parametric
and non-parametric Ziegler-Nichols tuning rules.\pagebreak{}
\end{abstract}

\section{Introduction}

This experiment broadly illustrates methods of modelling and better
understanding the behaviour of systems in both the discrete and continuous
time so that effective controllers maybe implemented. The aims of
the experiment can be summarised as follows:
\begin{enumerate}
\item Investigate the behaviour of various models for a flight including
limits of stability.
\item Develop an understanding of how the pilot behaves as a controller
and the effects of changing various parameters in the model for the
pilot on the behaviour of the system.
\item Developing a suitable autopilot for a flight using a Proportional-Integral-Derivative
(PID) controller and exploring limitations and methods of tuning of
this controller.
\end{enumerate}

\section{Experimental Work}

In this experiment, MATLAB was used to model transfer functions for
flights and a GUI was developed to allow the user to easily act as
the controller. Figure \ref{fig:GUI} shows the window that the user
interacts with. 
\begin{figure}[H]
\textit{\caption{A graphic of the GUI used in the experiment\label{fig:GUI}}
}
\centering{}\includegraphics[scale=0.4,bb = 0 0 200 100, draft, type=eps]{GUI.png}
\end{figure}
The aim of the controller is to keep the red line (which represents
the horizon) as close to the center as possible and definitely within
$\pm10$ green divisions from the centre position (failing this the
simulation ends). The actuator output, $u(t)$, is given by the position
of the user's mouse with the system responding to the change and the
position of the red line moving. Thus, the user's response is in continuous
time with the digital system in code responding at discrete-time intervals.
\\
The position of the user's mouse and the position of the red line
are plotted against time and the resulting graph is displayed. Modelling
the user as a proportional controller with a delay allows a bode plot
to be generated from which specific readings can be taken using the
Data Cursor function in MATLAB. This allows the stability margins
to be determined as well as other gain and phase values that can be
used in the analysis. Further detailed instructions for each section
of the experiment can be found in the lab handout attached as an Appendix.

\section{Theory}

The general system modelled in MATLAB, with the controller and plant,
can be seen in the Introduction of the lab handout which has been
attached as an Appendix. The closed-loop response is determined from
the open-loop response using tools such as Nyquist Diagrams and Bode
Plots. The analysis of the system via transfer functions is valid
because the system is linear and time-invariant. Intuitively, many
systems in reality are not linear but they can be linearised under
the right assumptions (for example $\sin\theta=\theta$ for small
$\theta$). This kind of linearisation has been performed in the generation
of the transfer functions used to generate the flight models. The
criteria a good controller satisfies is described in the Appendix.
\textbf{Further details of all control theory can be found in the
lab handout.}

\section{Results}

\textbf{Results for the experiment can be found in the figures contained
in the report attached as an Appendix.}

\section{Discussion}

\subsection{Sinusoidal Disturbances}

\begin{onehalfspace}
To calculate the complex value of $K(j\omega_{1})G(j\omega_{1})$,
the calculated values for gain and phase are taken from the Worksheet
in the Appendix and converted into the form $a+bi$. This gives a
complex value of $0.689+2.7447j$. Figure \ref{fig:Argand-Diagram}
shows the locus of points with stabilising gain as a solid line and
shows $|1+K(j\omega)G(j\omega)|$ for two valid points with dashed
lines. The value for $|1+K(j\omega_{1})G(j\omega_{1})|$ is $3.222$
which suggests that the closed loop will be attenuating. It follows
that the closed-loop will be attenuating for all $\omega$ since the
gain of the closed-loop transfer function is given by: 
\[
\frac{|K(j\omega)G(j\omega)|}{|1+K(j\omega)G(j\omega)|}
\]
 and a simple geometric analysis shows that $|1+K(j\omega)G(j\omega)|>|K(j\omega)G(j\omega)|$
for all valid $\omega$ and $K$.
\end{onehalfspace}

\begin{figure}[H]
\textit{\caption{Argand Diagram showing $K(j\omega_{1})G(j\omega_{1})$ and $|1+K(j\omega)G(j\omega)|$\textbf{\label{fig:Argand-Diagram}}}
}
\centering{}\includegraphics[scale=0.4,bb = 0 0 200 100, draft, type=eps]{sinatten.png}
\end{figure}
 

\subsection{Further analysis of the unstable aircraft}

Section 2.3 of the Appendix gives an unstable open loop transfer function:
\[
G_{2}(s)=\frac{2}{-1+sT}
\]
with a time delay, $D$ this transfer function becomes:

\[
G_{2}(s)=\frac{2e^{-Ds}}{-1+sT}
\]
For the closed loop to be stable, the Nyquist diagram should have
one encirclement of the point $(\frac{-1}{K},0j)$. From the transfer
function, the Nyquist plot can be generated for the two cases ($D>T$
and $T>D$). These are shown in Figure \ref{fig:Nyquist-plots}.

\begin{figure}
\caption{Nyquist plots for ($D>T$), left and ($T>D$), right\label{fig:Nyquist-plots}}
\centering{}\includegraphics[scale=0.75,bb = 0 0 200 100, draft, type=eps]{NyquistDT.png}
\end{figure}
A more analytical solution is to form the closed loop transfer function
and find the limit when the poles will be unstable. Thus the closed
loop transfer function is given by:
\[
G_{cl}(s)=\frac{2e^{-sD}}{-1+Ts+2e^{-Ds}}
\]
with poles satisfying $2e^{-Ds}-1+Ts=0$. The solution is difficult
to determine analytically but it is true that the limiting case is
when the line $y=1-Ts$ is tangent to the curve $2e^{-Ds}$. Enforcing
the equality of the first derivatives for tangency and solving for
$s$ gives:
\[
s_{tan}=\frac{1}{D}\ln\left(\frac{D}{T}\right)
\]
For the system to be unstable, $s_{tan}>0$. Which from the above
equation clearly gives that $D>T$.

\subsection{Broom Balancing}

Section 2.4 in the Appendix refers to finding the fastest unstable
pole that can be controlled for the unstable aircraft as being analagous
to finding the shortest length of broom that can be balanced. The
equations are given as follows:

\begin{align}
\ddot{x}+L\ddot{\theta} & =g\theta\label{eq:1}\\
x+L\theta & =y\label{eq:2}\\
y+T\dot{y} & =z\label{eq:3}
\end{align}
Equation \ref{eq:1} comes from the linearised dynamic equations of
the system, equation \ref{eq:2} from the measurement of the horizontal
displacement at the top of the rod and equation \ref{eq:3} from the
proportional, derivative feedback signal generated. Taking laplace
transforms and rearranging gives:

\begin{align}
\bar{\theta}(s) & =\bar{x}(s)\left(\frac{s^{2}}{g-Ls^{s}}\right)\label{eq:thetax}\\
\bar{y}(s) & =\bar{x}(s)+L\bar{\theta}(s)\label{eq:yxtheta}\\
\bar{z(s)} & =\bar{y}(1+Ts)\label{eq:zy}
\end{align}
where $T^{2}=L/g$. Now, putting (\ref{eq:thetax}) in (\ref{eq:yxtheta})
and then (\ref{eq:yxtheta}) into (\ref{eq:zy}) to eliminate $y$
and $\theta$ gives:
\begin{align*}
\bar{z}(s) & =\bar{x}(s)\frac{g(1+Ts)}{g-Ls}\\
G(s) & =\frac{1+Ts}{1-T^{2}s^{2}}\\
G(s) & =\frac{1}{1-Ts}
\end{align*}
This is just half of the negative value of the transfer function for
the unstable aircraft. Assuming this proportional gain can be accounted
for in the controller, the smallest value of T that can be stabilised
is given by the results obtained in the experiment for the unstable
aircraft i.e. $T=0.3s$. Since $T^{2}=L/g$, the shortest value of
$L$ that can be stabilised is 88.3cm. This seems like a reasonable
value as it is considered impressive online to balance a broom about
this length while sitting on a couch for around 19 minutes and has
not, as of yet, been beaten.\textsuperscript{\cite{key-16} }(This
source does not standardise the length of broom between competitors
as they have unfortunately not applied control theory to the problem).

\subsection{Ziegler-Nichols rules for tuning PID controllers}

The two principle methods of tuning are based on the two different
sets of parameters used to describe the system: 
\begin{enumerate}
\item Using the proportional gain at the point where the \textit{closed-loop}
system just oscillates and the time period of oscillations, stated
by Ziegler-Nichols as $S_{u}$ , the ``ultimate sensitivity'' and
$P_{u}$, the period of oscillation respectively.\textsuperscript{\cite{key-18} }If
the PID controller is described as in (3.1) in Section 3.1 of the
Appendix then the values for $K_{p}=0.6S_{u},\ T_{i}=0.5P_{u},\ T_{d}=0.125P_{u}$.
This method is described in literature as non-parametric.\textsuperscript{\cite{key-17}}
\item Perturbing the \textit{open-loop} system with a small impulse and
measuring the rate of response at the inflection point of the curve,
$R$ and the time delay in starting the response, $L$. These parameters
are better illustrated in Figure \ref{fig:Reaction-curve}. The optimal
values are then given as follows: $K_{p}=\frac{1.2}{R_{1}L},\ T_{i}=2L,\ T_{d}=0.5L$.
This method is described in literature as parametric.\textsuperscript{\cite{key-17}}
\begin{figure}[H]
\textit{\caption{Reaction curve taken from \cite{key-18} showing the physical meaning
of the parameters, R and L\label{fig:Reaction-curve}}
}
\centering{}\includegraphics[scale=0.6,bb = 0 0 200 100, draft, type=eps]{Screen Shot 2017-11-30 at 15.13.50.png}
\end{figure}
\end{enumerate}
These values attempt to provide a good load disturbance rejection
performance which means they often give poor damping ratios.\textsuperscript{\cite{key-17}}The
advantages and disadvantages of these two methods stems from the domain
in which the parameters for the system are determined. For a non-parametric
model, the parameters and tuning are done on the \textit{closed-loop}
so if the transfer function of the plant is not known, the tuning
will guarantee stability for a given gain margin under proportional
control only. Furthermore, more practically, the controller and plant
do not need to be disconnected in order to tune the controller. Some
response about the open-loop system is required for the parametric
method. If the open loop transfer function of the plant is well-defined
then the tuning factors can be determined more quickly from the transfer
function itself. This can be helpful when dealing with a system that
is not necessarily easy to access regularly for measurements. Thus
depending on the nature of the system (well-defined or not) the appropriate
tuning method should be used.

\subsection{Discretised PID controller}

The formula for calculating the integral and derivative were as follows:

\[
d=\frac{e_{\tau}-e_{\tau-1}}{T},\;i=i_{\tau-1}+e_{\tau}T
\]
taking Z-transforms yields:

\begin{align*}
\begin{alignedat}{1}D(z)= & \frac{E(z)-z^{-1}E(z)}{T}\\
\implies\frac{D(z)}{E(z)}= & \frac{1-z^{-1}}{T}
\end{alignedat}
\; & \begin{alignedat}{1}I(z)= & Iz^{-1}+E(z)T\\
\implies\frac{I(z)}{E(z)}= & \frac{T}{1-z^{-1}}
\end{alignedat}
\end{align*}
Therefore, to obtain the transfer function, plug $s=\frac{1-z^{-1}}{T}$
into the continuous-time transfer function to yield:
\[
G(z)=K_{p}\left(1+\frac{T}{T_{i}(1-z^{-1})}+T_{d}\left(\frac{1-z^{-1}}{T}\right)\right)
\]
This method of differentiation (and discretisation by algebraic substitution)
is called the backward Euler method\textsuperscript{\cite{key-15}}
and is always stable since Figure \ref{fig:BackEuler} shows that
it always maps within the unit-circle.
\begin{figure}[H]
\centering{}\textit{\caption{The locus of points given by the transfer funtion of the Backward
Euler method\label{fig:BackEuler}}
}\includegraphics[scale=0.6,bb = 0 0 200 100, draft, type=eps]{BackEuler.png}
\end{figure}
It is worth noting that had the integral been calculated using a different
method (trapezium rule) then the simple algebraic transformation would
not have sufficed in giving the discrete-time transfer function.

\subsection{Discretisation of blocks with time delays}

From the block diagram for discretisation given in Figure 2, page
10 of the lab handout in the Appendix, the following result is obtained:

\[
G(z)=\frac{1}{\bar{u}(z)}\mathcal{Z\left[L\mathrm{^{-1}}\left(\mathrm{\textrm{\ensuremath{\frac{e^{-D_{1}s}}{s}\bar{u}}(s)}}\right)_{\textrm{t=kT}}\right]}
\]

For a zero-order hold it is best to use input, $u(t)=1$. Taking the
laplace transform and Z-transform of this and plugging into the transfer
function, gives:
\[
G(z)=\frac{z^{-D_{1}}T}{z-1}
\]
and using the expansion of $D_{1}$ as $D_{0}+nT$ the following result
is obtained:

\[
G(z)=\frac{z^{-D_{0}}z^{-nT}T}{z-1}
\]
If the open-loop function is simply calculated excluding the time
delay and then added back in, the error will be of order $z^{-D_{0}}$
in the Z-domain. This suggests that if the time-delay is close to
an integer value of the sampling time, i.e. $D_{0}$ small, then both
methods yield good results. The smaller the sampling time, the better
the approximation as the term in the numerator has increasingly negligible
effect. For ease of calculation, adding an integer time delay is definitely
easier since there is limited precision required.

\subsection{Analysis of the PID controller with a lag}

Care must be taken when using \texttt{bodedisp.m} on the product of
the controller including the delay and the plant. Due to the nature
of a zero-hold, applying the same tuning paramters of a continuous-time
PID controller to a discrete-time PID controller means that effectively
a time delay of $T/2$. This means that \texttt{bodedisp.m }is likely
to have a greater error since $T/2$ is as far away from the integer
time period as possible! An additional delay in this case may also
cause instability. It is possible to modify this by analysing the
system in state-space or by modifying \texttt{bodedisp.m }so that
a greater precision is used (i.e. more resolution than an integer
time period).

\section{Conclusions}
\begin{enumerate}
\item User flight control can be appropriately modelled as a proportional
gain with a time delay.
\item The stability of a single unstable pole system, is compromised if
a delay is introduced that is greater than the time period of the
pole.
\item Similar analysis of one control system can be used to predict the
behaviour of another if they both have similar transfer functions.
\item Additional considerations are required when using a PID contoller
such as the saturation of actuator signals that occur from limitations
in reality.
\item Ziegler-Nichols tuning rules can be used in two forms with advantages
based on the nature of the control system being described.
\item Discretisation of time-delay systems can be difficult due to the interaction
between the sampling time and the time delay.
\item PID controllers tuned to time delay systems in continuous time behave
differently when discretised and choosing a suitable sampling time
can reduce the severity of such problems.
\end{enumerate}
\begin{thebibliography}{1}
\bibitem{key-15}Forni F. 2017, lecture notes distributed in 3F1:
Signals and Systems, Lecture 9: Infinite Impulse Response Filters.
Available from: https://www.vle.cam.ac.uk/

\bibitem{key-16}McManaman D. 2011. \textit{Longest Time To Balance
Broom On Hand While Sitting On Couch }{[}Online{]}. Record Setter.
Available at: https://recordsetter.com/world-record/balance-broom-hand-while-sitting-couch/9156.
{[}Accessed 30/11/2017{]}.

\bibitem{key-18}Ziegler, J.G. and Nichols, N.B., 1942. \textit{Optimum
settings for automatic controllers}. trans. ASME, 64(11). {[}Accessed
Online 30/11/2017{]}

\bibitem{key-17}Visioli, A., 2010. \textit{Control of Integral Processes
with Dead Time}. Springer, London {[}Accessed Online 30/11/2017{]}
\end{thebibliography}

\end{document}
